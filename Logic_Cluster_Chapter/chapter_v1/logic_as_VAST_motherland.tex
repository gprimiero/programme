\documentclass{article}

%% Make these package calls in any driver file or sub-file commands.tex:
\usepackage{harvard}
\usepackage{amsfonts}
\usepackage{mathrsfs}
\usepackage{fancybox}

\begin{document}



\section{The Motherland is Vast; The Colony, Hitherto, Tiny}
\label{sect:motherlandvast_colonytiny}

\noindent
%
Once Logic is taken to be the Motherland, and programs as most humans
have so far created and used them taken to be a Colony, say
$C_\textsc{<h}$, an immediate moral from the metaphor is that the
Motherland is breathtakingly larger and richer than what has been seen
in $C_\textsc{<h}$.  Put another way, the colony $C_\textsc{<h}$ is
tiny; but the Motherland is vast, with glorious riches and raw
materials entirely absent from, and not even faintly reflected by
anything in, $C_\textsc{<h}$.  What has been introduced as Pure
General Logic Programming (PGLP) \cite{introduction_of_PGLP} makes
this clear; the tiny-vs.-vast moral also follows from the view,
defended in \cite{comp_sci_immaterial_formal_logic}, that past and
present programming, and indeed computer science overall, is a small
fragment of formal logic.

These are abstract concepts, yet even simple examples can be used to
reveal the moral brought to the reader's attention in the present
section.  For example, where the background formal language is an
infinitary one for the standard propositional calculus that permits
countably infinite disjunctions and conjunctions\footnote{The formal
  language here is the propositional fragment of that for the
  well-behaved infinitary logic $\mathscr{L}_{\omega_1 \omega}$,
  economically covered in \cite{ebb.flum.thomas.2nded}.}, here is a
two-line program $\mathbb{P}_1$ in PGLP from the Motherland, hereby
brought to $C_\textsc{<h}$:

\medskip
$$
\begin{array}{l|l}
 1 &  \phi_1 \vee \phi_2 \vee \phi_3 \ldots\\
  2 & \neg \phi_2 \wedge \neg \phi_3 \wedge \neg \phi_4 \ldots 
\end{array}
$$
\medskip
                    
\noindent
%
The query ``$\phi_1$?'' against $\mathbb{P}_1$ then results in an
affirmative answer, and the proof makes use of an infinitary analogue
of \textit{disjunctive syllogism} or \textit{unit resolution}.









\bibliographystyle{agsm}
\bibliography{main72}

\end{document}
