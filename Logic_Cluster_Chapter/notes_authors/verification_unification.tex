\documentclass[11pt]{article}

%% Package calls:
  \usepackage{harvard}
  \usepackage{eufrak}
  \usepackage{mathrsfs}
  \usepackage{mathtools}
%%  \usepackage{hyperref}


\begin{document}


  \title{\textbf{Serious Verification as\\
                 the Unifier of Logicist Programming}}
  \author{RT $\leftrightarrow$? SB $\bullet$ GP $\bullet$ ED $\bullet$ J-BJ?}
  \date{\texttt{rough non-detechnicalized version 0911191139NY}}
  \maketitle


  \begin{abstract}
    \noindent
%
      After distinguishing between two fundamentally different
      conceptions of logicist (computer) programs and programming (the
      declarative vs.\ the constructivist traditions), we consider the
      challenge of what we call `serious' program verification, and
      find that it catalyzes the unification of these standardly
      competing conceptions.  The unification can be viewed as
      offering to others an integrated logicist paradigm, cast at a
      philosophical level, for subsequently achieveing high-assurance
      computer programming.
  \end{abstract}

\thispagestyle{empty}





%%%%%%%%%%%%%%%%%%%%%%%%%%%%%%%%%%%%%%%%%%%%%%%%%%%%%%%%%%%%%%%%%%%%%%
\newpage
\thispagestyle{empty}

  \begin{small}
    \tableofcontents
  \end{small}






%%%%%%%%%%%%%%%%%%%%%%%%%%%%%%%%%%%%%%%%%%%%%%%%%%%%%%%%%%%%%%%%%%%%%%
  \newpage
  \pagenumbering{arabic}



  \section{Introduction}
  \label{sect:intro}
 
    After distinguishing between two fundamentally different
    conceptions of logicist (computer) programs and programming (the
    \textbf{declarative} vs.\ the \textbf{constructivist} traditions),
    we consider the challenge of what we call `serious' program
    verification, and find that it catalyzes the unification of these
    ordinarily competing conceptions.  The unification can be viewed
    as offering to others an integrated logicist paradigm, cast at the
    philosophical level, for subsequently achieveing high-assurance
    computer programming.

    The plan for the paper is as follows.  We begin (\S
    \ref{sect:basic_declarative}) with a synopsis of basic declarative
    programming, including a few remarks about its historical roots
    and current real-world practice.  Next (\S
    \ref{sect:ver_challenge_dec}), we consider an approach to program
    verification on the declarative approach, and find that initial
    sanguinity regarding this approach melts away to reveal that what
    can be called `serious' verification is outside the reach of this
    approach.  The upshot is a turning to the constructivist approach
    for rescue; section \ref{sect:basic_constructivist} sets out in
    broad terms this approach, and then section \ref{sect:rescue}
    explains the rescue itself.  A brief concluding section wraps up
    the paper, and includes some remarks regarding the future of the
    unified approach we have introduced.



  \section{Basic Declarative Logicist Programming}
  \label{sect:basic_declarative}

  \noindent
%
  Let $\Phi$ be a set of wffs constructed in accordance with a formal
  grammar of an extensional logic $\mathscr{L}$ sufficiently less
  expressive than first-order logic $\mathscr{L}_1$ to avoid the
  \textit{Entscheidungsproblem}.\footnote{E.g.\ $\mathscr{L}$,
    formula-wise, can be restricted to those having at most two
    distinct variables, and allowing identity.  Such a logic is
    decidable (as shown by Scott, who built on G\"{o}del's
    completeness theorem, and --- for the allowance of identiy ---
    Mortimer after Scott), and dodges Church's Theorem.} We say that
  $\Phi$ is a \textbf{logic program}.  Our logic $\mathscr{L}$
  includes some collection $\mathcal{I}^\ast_\mathscr{L}$ of inference
  schemata for deduction that gives the proof theory of this logic,
  and when some formula $\phi$ is deducible from $\Phi$ in this theory
  we --- following standard notation --- write
  $$\Phi\ \vdash_{\mathcal{I}^\ast_\mathscr{L}}^Y \phi.$$
%
  If instead of `Y' in this statement there is an `N,' then what is
  being said is that deducibility fails to hold.  When the provability
  relation symbol is superscripted as in
  $$\Phi\ \vdash_{\mathcal{I}^\ast_\mathscr{L}}^\pi \phi,$$
%
  what is being said is that particular proof $\pi$ goes from $\Phi$
  to $\phi$ in the relevant proof theory.  To say that it's a question
  and at present unknown as to whether there exists a proof from
  $\Phi$ to $\phi$ by the inference schemata in question, we write
  $$\Phi\ \vdash^?_{\mathcal{I}^\ast_\mathscr{L}}\ \phi.$$
%

  We have said that $\Phi$ is a program.  Surely that will seem to
  most to be rather odd.  After all, this means that a mere collection
  of formulae is counted as a program, and such a collection seems
  rather static.  How does something happen?  After all, surely the
  point of a program is to set matters up so that something meaningful
  can happen under the governance of said program.  In the present
  context, things begin to happen when, first, we issue a query
  $\mathfrak{q}$ against the program $\Phi$.  For example, set
  $\Phi'\ \coloneqq \{p \rightarrow q, q \rightarrow r, \neg r\}$, and
  consider a query as to whether from $\Phi'$ is can be proved that
  $\neg p$.  The query is given to a reasoner $\mathcal{R}$, and in
  this case, of course, the answer given back will be a Yes,
  accompanied by a simple proof that we shall label $\pi'$; this proof
  might for instance first use hypothetical syllogism to obtain $p
  \rightarrow r$, and from this and given $\neg r$ by \textit{modus
    tollens} deduce $\neg p$.  In the notation introduced above, we
  have the following query in the example now before us:
  $$\mathfrak{q} \coloneqq \Phi'\ \vdash^?_{\mathcal{I}^\ast_\mathscr{L}}\ \neg p$$
%
  And we have have the following returned by automated reasoner $\mathcal{R}$:
$$\Phi'\ \vdash^{\pi'}_{\mathcal{I}^\ast_\mathscr{L}}\ \neg p.$$

  For more in the direction sketched above, see
  \cite{comp_sci_immaterial_formal_logic}.


  \section{The Verification Challenge to the Declarative Approach}
  \label{sect:ver_challenge_dec}

    Now, how do we carry out verification of the program and of the
    result of the reasoner taking it and producing its output?  The
    immediate and sanguine response to this question is that we have
    only to verify that the returned $\pi$ is a valid proof, all the
    way through, that its starting givens match $\Phi$, and that its
    ultimate conclusion matches $\phi$.  This can be accomplished by a
    dirt-simple automated proof checker $\mathcal{C}$.\footnote{Note
      that while in our example the proof $\pi'$ can be checked by
      hand, nearly all proofs will be too long to be manually checked,
      and programs will be executed repeatedly through time for any
      real-world applications, which puts manual checking further out
      of the realm of feasibility.}

    But this is absolutely \emph{not} a meeting of the challenge of
    \emph{serious} program verification.  In serious program
    verification all elements of the overall process of computing must
    be scrutinized, top to bottom.  From this perspective, what is
    left out?  What gets a free pass?  The answer is obvious: the
    checker has been ignored.  Why should we trust $\mathcal{C}$?  By
    definition, it is outside the paradigm that has been set out!



  \section{Basic Constructivist Logicist Programming}
  \label{sect:basic_constructivist}



  \section{Constructivists to the Rescue}
  \label{sect:rescue}



  \section{Conclusion}
  \label{sect:conclusion}



  \section{Acknowledgements}
  \label{sect:ack}
 
  This paper was is an outcome of the ANR PROGRAMme project
  (ANR-17-CE38-0003-01), specifically of interaction at the Bertinoro
  2018 meeting.

  

  \bibliographystyle{agsm}
  \begin{small}
    \bibliography{main72}
  \end{small}


\end{document}

