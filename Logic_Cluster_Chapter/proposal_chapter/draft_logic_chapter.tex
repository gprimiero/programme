\documentclass[]{article}

%opening
\title{What is a (Logic) Program}
\author{}

\begin{document}

\maketitle

\begin{abstract}

\end{abstract}

\section{Introduction}

the chapter moves roughly around the divide ``constructive vs. classical'' semantics of programs.\\


(JB) proposal to starts from the Strachey-Scott view / denotational semantics as a way to ground the different definitons of computation. semantics as a way of distinguishing between the two paradigms (connection to languages group in the volume)


CH isomorphims:
-) Introduction to pure lambda-calculus as an abstract functional programming language;
-) Possibility of defining “pathological” programs in pure lambda-calculus (like looping programs or never ending ones);
-) Use of a type system based on the logical implicational fragment in order to get always terminating programs (i.e. total functions). Correspondence between simple typed lambda-calculus and the implicational fragment in natural deduction;
-) Extension of the Curry-Howard correspondence to first-order logic and to arithmetic. Use of new functional programming instruction in order to decorate the rules of natural deduction. 
-) Introduction of dependent type systems and extension of Curry-Howard correspondence to second order logic. This could be interesting to show that some of the “pathological” programs of pure lambda-calculus can now be typed (in the Girard-Reynolds system F), e.g. the program (x x).




operational vs. denotational interpretation

typing vs. realizability interpretation

natural deduction
intuitionistic logic
constructive tt


Edgar look at actors

\section{philosophical questions}

\begin{enumerate}
\item intensional vs extensional?: Is there a way to identify programs on an extensional way (i.e. by an equivalence relation) Or is the notion of program a purely intensional one?

\item CTT?

\item conceptual order between computation and logic

\item invariants: what are the abstract conditions on a given information dynamics for meaning to emerge

\item pathological vs. correctness 
\end{enumerate}



\end{document}
