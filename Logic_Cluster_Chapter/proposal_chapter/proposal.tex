\documentclass[]{article}

%opening
\title{Logic and Programs:\\
	 a Declaration of (In)dependence}
\author{Nicola Angius, Selmer Bringsjord, Edgar Daylight,\\
	 Liesbeth de Mol, Alberto Naibo, Giuseppe Primiero,\\
	  Raymond Turner, Nick Wiggerhaus}

\begin{document}

\maketitle

%\begin{abstract}
%
%\end{abstract}

\section{What is this chapter about: Bertinoro as our Hall of (In)dependence.}

How does our Community understand the relation between Logic and Programs: Motherland and its Colonies, or Rebels against the Emperor?\\ 

The core problem is the choice between the rule of law and the freedom of practice: how much control does a program need? How much can a physical program give back to its logic? Is there a choice to be made?
Or can we reach a balanced, fruitful, "special relationship"? What does that mean for the design of logical systems, programming languages and their implementations?\\

In this chapter we will try to explore and understands all the problems and the opportunities of colonization.



\section{Topics and problems around Logic and Programs}

The list of topics emerging from the individual contributions, and which could be seen as sections:

\begin{enumerate}

\item \textit{Constitution}: logic as specification 
\item \textit{Independence}: relation model-implementation
\item \textit{Representative Democracy}: program vs. model of programs
\item \textit{The Parties}: declarative and constructivist traditions

\item \textit{Controlled behaviours}: types in logic and in programming

\item \textit{Magna Carta}: distinct linguistic levels of formulations for the same program
\item \textit{Habitus}: notion of logic contextual on practices
%http://criticallegalthinking.com/2019/08/06/pierre-bourdieu-habitus/
\item \textit{Loopholes in the Constitution}: perfection/limitation vs. approximation/improvement
\item \textit{Transparency and Dialogue}: another role for formal methods
%: opacity and communication gaps
%- the what (specification) and the how (implementation), (in my terminology the blueprint and the design)-- see programming paradigm


\end{enumerate}

%%--- Mathematical definitions of functions are to be read axiomatically and provide no mechanism of evalua-
%tion. Lazy evaluation allows the language run-time to discard sub-expressions
%that are not directly linked to the final result of the expression. It reduces the
%time complexity of an algorithm by discarding the temporary computations and
%conditionals etc. etc. None of this concerns the specification of the intended
%function which is defintional and axiomatically given.
%
%
%--- Kowalski at a special meeting of the Royal Society in London emphasises
%the relationship between logic programs and specifications. He observe that
%the only difference between a complete specification and a program is one of
%efficiency.
%
%
%-- complexity and evaluation as criteria distinguish specification and implementation
%
%-- other level: physical execution


\section{Issues in preparing this chapter}


The main issue in preparing this chapter is necessarily that we do not share the same view, and we cannot possibly hope to present a position paper. But, it seems, we do share the same problem. Moreover, the editors wants us to provide something different to the literature: not another paper that present some approach and some solution to a problem.\\

We will need to collaborative and find the narrative and the style. What I have proposed above is a conceptual reading of the problem that arises in all short abstract that were sent to me: colonization and independence as the duality that regulates the life of the Empire and the Colonies. The way to proceed is unclear: we might assign the sections specified above to one or more authors (based on their proposals) and then mold everything together with a unified approach.\\

I am open to objections, criticisms and firm oppositions to the project in this form. We are in a representative democracy, bordering a direct democracy. I am sorry for the constitutional monarchists.

\section{Connections to other Chapters}

\begin{enumerate}

\item Machines
\item ?
\end{enumerate}


\section{Ideas for possible Programs to start with}


\begin{itemize}

\item A program written in Curry's notation
\item A Coq program
\end{itemize}

These two examples share a lot in common: logical and physical, control, notation, correctness. Other examples welcome.

\end{document}
